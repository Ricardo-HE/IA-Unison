\documentclass[onecolumn, letter, 11pt]{article}

\usepackage[latin1]{inputenc}
\usepackage[sumlimits]{amsmath}
\usepackage{graphicx}
\usepackage[spanish]{babel}
\usepackage{amsfonts}
\usepackage{amssymb}
\usepackage{amsthm}
\usepackage{geometry}

\geometry{left=2.5cm,right=2.5cm,top=2cm,bottom=2cm}


\title{Examen r�pido No. 1}
\author{\textbf{Inteligencia Artificial}\\Julio Waissman Vilanova}
\date{\today}


\begin{document}

\maketitle

\section{Tipos mutables}

\begin{enumerate}
\item Mira el siguiente c�digo y ejecutalo:
\begin{verbatim}
# Vamos a generar 3 listas que parecer�an iguales
lista_1 = [1, 2, 3, "toto", ["a", "b", "c"]]
lista_2 = lista_1
lista_3 = lista_1[:]

lista_2[0] = "XXX"
print(lista_2)

lista_3[-1][0] = 1000
print(lista_3)
\end{verbatim}

  Ahora responde (sin revisar) lo que crees que deber�a salir con
\begin{verbatim}
print(lista_1)
print(lista_2)

# Ejemplo:
# print(lista_3)  -->  [1, 2, 3, 'toto', [1000, 'b', 'c']]
\end{verbatim}
  Una vez que lo contestaste, ejecuta las operaciones y explica el
  porqu� de los resultados.

\item Revisa el siguiente c�digo
\begin{verbatim}
class coordenadas:
    def __init__(self, x=0, y=0):
        "Inicializa un objet coordenadas"
        self.x = x
        self.y = y

    def __str__(self):
        return "\n\tCoordenada x = {}\n\tCoordenada y = {}".format(self.x, self.y)


a = coordenadas(3, 4)
b = a
b.x = 100
print("b = " + str(b))
\end{verbatim}
  Y ahora escribe que crees que deber�a salir (sin revisar) con
\begin{verbatim}
print("a = " + str(a))
\end{verbatim}
Una vez que lo contestaste, revisa el resultado y expl�calo brevemente.
  
\end{enumerate}

\section{Listas y diccionarios}

\begin{enumerate}
\item Escribe, en una sola linea, una expresi�n que genere todos los
  n�meros enteros que se encuentran entre $1$ y $1000$ que sean
  divisibles por $2$, $3$, $5$ y $7$ al mismo tiempo. Aprovecha de las
  ventajas para crear listas de:
\begin{verbatim}
	[expresi�n  for elemento in lista],
	[expresi�n  for elemento in lista if condici�n]
\end{verbatim}
 
\item Escribe una funci�n que reciba una lista de elementos (letras,
  numeros, lo que sea), cuente la ocurrencia de cada elemento en la
  lista y la devuelva en forma de diccionario e imprima un histograma
  de ocurrencias, por ejemplo:
\begin{verbatim}
>>> d = FuncionEjemplo( [1,'a',1, 13, 'hola', 'a', 1, 1, 'a', 1], Imprime = True)

1               *****   (5 -> 50%)
'a'             ***     (3 -> 30%)
13              *       (1 -> 10%)
'hola'          *       (1 -> 10%)

>>> print d
{1:5, 'a':3, 13:1, 'hola':1}		
\end{verbatim}

\item Escribe una funci�n que modifique un diccionario y regrese el
  diccionario modificado y una copia del original, donde cada entrada
  del diccionario sea una lista de valores, ejemplo de la funci�n:
\begin{verbatim}
>>> dic1 = {'Pepe':[12, 'enero', 1980], 'Carolina':[15,'mayo',1975],'Paco':[10,'nov',1970]}
>>> dic2 = fundicos(dic1, 'Pepe', 1, 'febrero')
>>> print dic1 
{'Pepe':[12, 'enero', 1980], 'Carolina':[15,'mayo',1975],'Paco':[10,'nov',1970]}
>>> print dic2 
{'Pepe':[12, 'febrero', 1980], 'Carolina':[15,'mayo',1975],'Paco':[10,'nov',1970]}
\end{verbatim}

\end{enumerate}

\section{Funciones y clases}

\begin{enumerate}

\item Escribe una funci�n \texttt{ fun1 } que reciba un n�mero $n$ y
  calcule el n�mero primo inmediatamente superior. Escribe una funci�n
  \texttt{ fun2 } que reciba como argumento un numero y una funci�n, y
  devuelva una lista con la evaluaci�n de la funci�n desde $1$ hasta
  $n$. Prueba la funci�n con \texttt{fun1} y con \texttt{math.sqrt}.

\item Escribe una funci�n, lo m�s compacta posible, que escoja entre
  los 3 patrones ascii a continuaci�n, e imprima en pantalla el
  deseado, pero de la dimensi�n $n$ deseada ($n \ge 4$, toma en cuanta
  que para algunos valores de $n$ habr� alg�n(os) patrones que no se
  puedan hacer).
\begin{verbatim}
          *             ++++           oooooooo
          **            ++++           ooo  ooo
          ***           ++++           oo    oo
          ****          ++++           o      o
          *****             ++++       o      o
          ******            ++++       oo    oo
          *******           ++++       ooo  ooo
          ********          ++++       oooooooo
\end{verbatim}


\item Dise�a una clase Matriz e implementa con sobrecarga la suma de
  matrices, la multiplicaci�n de matrices y la multiplicaci�n por un
  escalar, eliminar columna y eliminar fila. Como inicializaci�n de un
  objeto es necesario conocer $n$ y $m$ (en caso de no proporcionarlos
  la matriz tendr� una dimensi�n de $1 \times 1$. Igualmente, de no
  especificarse todos los elementos se inicializan a 0, a menos que
  exista un tipo espacial ( 'unos' o 'diag' por el momento). Programa
  la representaci�n visual de la matriz. Ten en cuenta tambien el
  manejo de errores. Por ejemplo, para su uso:
\begin{verbatim}
>>> A = Matriz(n=3, m=4)
>>> print A
0 0 0 0
0 0 0 0
0 0 0 0
>>> A = A.quitafila(2)
>>> print A
0 0 0 0
0 0 0 0
>>> B = Matriz(4,4,'diag')
>>> print B
1 0 0 0
0 1 0 0
0 0 1 0
0 0 0 1
>>> C = Matriz(4,1,'unos')
>>> print C
1
1
1
1
>>> D = 3*B*C
>>> print D
3
3
3
3
>>> E = 3*B + C
error "Si no son de la misma dimensi�n las matrices no se pueden sumar"
\end{verbatim}
\end{enumerate}
\end{document}
